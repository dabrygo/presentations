\documentclass{article}

\usepackage{url}

\begin{document}
  \begin{enumerate}
    \item \textit{Names}

    Style aside, which of these are valid Python (variable, class, module, etc.) names?
    \begin{itemize}
      \item MY\_VARIABLE
      \item 99\_bottles\_of\_pop
      \item c3p0
      \item r2-d2
      \item BrOKeN\_SHIft\_kEY
      \item Total
      \item \$VAR
      \item \_ (note: a single underscore)
      \item \_var\_
      \item \_\_var
      \item \_\_var\_\_
      \item \_\_\_\_\_\_\_\_
    \end{itemize}

    \textsc{Idea}: Try to guess on your own, then use \texttt{str.isidentifier()}
    to check!

    \item \textit{Padding/Formatting, Exponentiation}

    Re-create the following table:
    \begin{verbatim}
      n  | n**2
     ----------
      0  |   0
      1  |   1
      2  |   4
      3  |   9
      4  |  16
      5  |  25
      6  |  36
      7  |  49
      8  |  64
      9  |  81
    \end{verbatim}

    \item \textit{Slicing/Concatenation}

    For a key of 3, Rail-fence Cipher has ``posts" of 
    evenly-spaced letters. Encrypt

    \begin{center}
      \texttt{defend the east wall of the castle}
    \end{center}

    as 

    \begin{center}
      \texttt{dnetlhseedheswloteateftaafcl}
    \end{center}


    as shown on \url{practicalcryptography.com/ciphers/classical-era/rail-fence}.
    Note that they ignore spaces! 
    
    It's ok to ``hardcode'' values. We only want the encryption of this
    particular message for this particular key. But if you can generalize, do
    it!

    \textsc{addendum}: Concatenation (not covered in lecture) can be done
    by 
    
    \begin{center}
      \texttt{`hello' + `world'}
    \end{center}
   
    \item \texttt{math.floor}, \textit{modulus/remainders}
    
    Programmatically print out the decrypted Rail-fence Cipher 
    \begin{verbatim}
    I . . . A . . . _ . . . E . . . Z . . . S
    . _ . E . L . Y . L . K . _ . U . Z . E .
    . . R . . . L . . . I . . . P . . . L . .
    \end{verbatim}
    by following Wikipedia's instructions
    at \url{https://en.wikipedia.org/wiki/Rail_fence_cipher#Solution}. 
    \textbf{Bonus}: Use your program to decrypt last problem's code.

    \textsc{addendum}: We can convert a string to a list to make lines like the
    above:

    \begin{center}
    \texttt{`!' + `..'.join(list(`abc')) + `?'}
    \end{center}

    yields

    \begin{center}
    \texttt{`!a..b..c?'}
    \end{center}

    \item \texttt{ord} / \texttt{chr}

    \texttt{ord} was used to reduce the characters in this message to their
    code points.  Use \texttt{chr} to decrypt the message!

    65 32 77 248 248 115 101 32 111 110 99 101 32 98 105 116 32 109 121 32 115 105 115 116 101 114 46 46 46 32 78 111 32 114 101 97 108 108 105 33


    \item \texttt{math}

    Complete the function:
    \begin{verbatim}
    def law_of_cosines(a, b, gamma): 
        return 0
    \end{verbatim}
    
    generalizing \texttt{math.hypot} for \textit{any} triangle with the
    \textsc{Law of Cosines}:

    \[
      c^2 = a^2 + b^2 - 2ab\cos\gamma
    \]
    where $\gamma$ is the angle between sides $a$ and $b$.

  \end{enumerate}
\end{document}
