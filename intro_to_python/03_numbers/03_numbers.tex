\documentclass{beamer}

\usepackage{color}
\usepackage{graphicx}
\usepackage{hyperref}
\usepackage{listings}
\usepackage{minted}


\title{IntroPy}
\subtitle{Numbers}

\begin{document}
\frame{\titlepage}

\begin{frame}
  \frametitle{The Three Basic Numeric Types}

  \begin{table}
    \begin{tabular}{l | l | l}
    Function & Use         & Output   \\
    \hline
    \texttt{int}      & 123         & 123      \\
                      & int('42')   & 42       \\
                      & int(3.1415) & 3        \\
    % TODO Base
    \hline
    \texttt{float}    & 123.0         & 123.0   \\
                      & float('3.14') & 3.14    \\
                      & float(42)     & 42.0    \\
    \hline
    \texttt{complex}  & 1j                      & 1j    \\
                      & 1 + 2j                  & (1+2j)\\
                      & complex(real=0)         & 0j    \\
                      & complex(real=1)         & (1+0j)\\
                      & complex(imag=1)         & 1j    \\
                      & complex(imag=2, real=1) & (1+2j)\\
                      & complex(1, 2)           & (1+2j)
     \end{tabular}
  \end{table}
  
\end{frame}


% TODO Mention string formatting as base?
\begin{frame}
  \frametitle{Base}

  \begin{itemize}
    \item \texttt{hex}
    \item \texttt{oct}
  \end{itemize}
\end{frame}

\end{document}

% Math / operations
%  * Boolean (and, or, not)
%    - Operate booleans and numbers (except not)
%    - short-circuiting with `x or y`
%  * Set operations
%  * Comparison operators 
%    - Chainable
%  * Membership (``in'')
%  * `+/*` on lists
%  * `**`
%  * //
%  * augmented assignment
%    - pitfall: list += string
%  * math
%  * complex
%  * cmath
%  * numpy
%  * scipy
%  * statistics 
%  * int
%  * float
%  * matrices?
%  * rounding / Decimal
%  * Fractions
%  * Range
%  * chr / ord

