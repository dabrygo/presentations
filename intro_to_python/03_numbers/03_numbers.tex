\documentclass{beamer}

\usepackage{color}
\usepackage{graphicx}
\usepackage{hyperref}
\usepackage{listings}
\usepackage{minted}


\title{IntroPy}
\subtitle{Numbers}

\begin{document}
\frame{\titlepage}

\begin{frame}
  \frametitle{The Three Basic Numeric Types}

  \begin{table}
    \begin{tabular}{l | l | l}
    Function & Use         & Output   \\
    \hline
    \texttt{int}      & 123         & 123      \\
                      & 1\_353\_257 & 1353257  \\
                      & int('42')   & 42       \\
                      & int(3.1415) & 3        \\
    % TODO Base
    \hline
    \texttt{float}    & 123.0         & 123.0   \\
		      & 3.14\_15\_92  & 3.141592\\
                      & float('3.14') & 3.14    \\
                      & float(42)     & 42.0    \\
    \hline
    \texttt{complex}  & 1j                      & 1j    \\
                      & 1 + 2j                  & (1+2j)\\
                      & complex(real=0)         & 0j    \\
                      & complex(real=1)         & (1+0j)\\
                      & complex(imag=1)         & 1j    \\
                      & complex(imag=2, real=1) & (1+2j)\\
                      & complex(1, 2)           & (1+2j)
     \end{tabular}
  \end{table}
  
\end{frame}


\begin{frame}
  \frametitle{From Decimal to Base}
  \textbf{Integers}
  \begin{table}
    \begin{tabular}{l | l}
    Use         & Output   \\
    \hline
    bin(42)     & `0b101010' \\
    oct(42)     & `0o52'     \\
    hex(42)     & `0x2a'     \\
    \end{tabular}
  \end{table}

  \textbf{Floating Point}
  \begin{table}
    \begin{tabular}{l | l}
    Use          & Output   \\
    \hline
    1.0.hex()    & `0x1.0000000000000p+0'  \\
    (-1.0).hex() & `-0x1.0000000000000p+0' \\
    3.1415.hex() & `0x1.921cac083126fp+1'
    \end{tabular}
  \end{table}
\end{frame}


% TODO Mention string formatting as base?
\begin{frame}
  \frametitle{From Base to Decimal (Integer)}
  
  \begin{table}
    \begin{tabular}{l | l }
      Use                     & Output   \\
      \hline
      int(x=`42')          & 42          \\
      int(x=`42', base=10) & 42          \\
      \hline
      int(x=`10\_1010', base=2) & 42     \\
      int(x=`0b101010', base=2) & 42     \\
      \hline
      int(x=`0o71', base=8) & 57         \\
      int(x=`0o71', base=8) & 57         \\
      \hline
      int(x=`BaDC0fFee', base=16) & 50159747054\\
      int(x=`0xBaD\_C0fFee', base=16) & 50159747054\\
      \hline
      int(x=`python', base=36) & 1570137287\\
      int(x=`0', base=1) & ValueError \\
      \end{tabular}
  \end{table}
\end{frame}

\begin{frame}
  \frametitle{From Base to Decimal (Floating Point)}

  \begin{table}
    \begin{tabular}{l | l}
    Use          & Output   \\
    \hline
    float.fromhex(`0x1.0000000000000p+0')    &  1.0    \\
    float.fromhex(`-0x1.0000000000000p+0')   & -1.0    \\
    float.fromhex(`0x1.921cac083126fp+1')    &  3.1415 \\
    \end{tabular}
  \end{table}
\end{frame}

\begin{frame}
  \frametitle{Arithmetic Operators}

  \begin{enumerate}
  \item Four-function math $+, -, *, /$
  \item $/$ defaults to float division, even when given integers (e.g. $1/2=0.5$)
  \item $//$ does ``integer division'', truncates decimal (e.g. $9//10=0$)
  \item $\%$ (modulus) is remainder after division (e.g. $5 \% 3 = 2$)
  \item $**$ is exponentiation (e.g. $2**3=8$)
  \end{enumerate}
\end{frame}

\begin{frame}
  \frametitle{Builtins}

  \begin{enumerate}
  \item \texttt{abs(x)} is absolute value of $x$
  \item \texttt{divmod(x, y)} returns $(x//y, x \% y)$
  \item \texttt{pow} is exponentiation with optional modulus
    \begin{enumerate}
      \item $pow(2, 3) = 8$
      \item $pow(5, 2, 10) = 5$ (the last digit of $5**2=25$)
    \end{enumerate}
  \item \texttt{round} rounds a number to specified digit
    \begin{enumerate}
      \item $round(123.455) = 123.0$
      \item $round(123.455, ndigits=2) = 123.45$
      \item $round(123.455, ndigits=-1) = 120.0$
    \end{enumerate}
  \end{enumerate}
\end{frame}


\end{document}

% Math / operations
%  * Boolean (and, or, not)
%    - Operate booleans and numbers (except not)
%    - short-circuiting with `x or y`
%  * Set operations
%  * Comparison operators 
%    - Chainable
%  * `**`
%  * //
%  * math
%  * complex
%  * cmath
%  * numpy
%  * scipy
%  * statistics 
%  * int
%  * float
%  * matrices?
%  * rounding / Decimal
%  * Fractions
%  * Range
%  * chr / ord
%  * pint

