\documentclass{beamer}

\usepackage{color}
\usepackage{graphicx}
\usepackage{hyperref}
\usepackage{listings}
\usepackage{minted}


\title{IntroPy}
\subtitle{Strings}


\begin{document}
\frame{\titlepage}

%## Strings
%  * API
% * Formatting
%   * `%`
%   * `.format()`
%   * `f'{}'`
%   * template?
%   * Numbers
% * b''
% * r''
% * Unicode
% * str

%
% Properties
%
\begin{frame}
  \frametitle{Properties}
\end{frame}


%
% Making a String
%
\begin{frame}
  \frametitle{Making a String}
  \begin{itemize}
    \item Single- or double-quotes
      \begin{itemize}
        \item \lstinline{'hello world'} 
	\item \lstinline{"hello world"}
      \end{itemize}

    \item Three-quote pairs for multiline text
%    \inputminted{python}{multiline.py}

    \item Special characters start with \textbackslash, e.g.: 
      \begin{itemize}
        \item ``\textbackslash n'' (newline)
	\item ``\textbackslash t'' (tab) 
      \end{itemize}

    \item \textbackslash \textbackslash~ escapes backslash 
      \begin{itemize}
        \item \lstinline{'C:\\\\Users'}
      \end{itemize}

 \end{itemize}
\end{frame}


%
% Capitalization TODO add casefold and islower/title/upper/
%
\begin{frame}
  \frametitle{Case}
  Given \lstinline{car = 'Lightning McQueen from Cars'}
  \begin{table}
    \begin{tabular}{c | c}
    Function & Output \\
    \hline
    car.upper() & 'LIGHTNING MCQUEEN FROM CARS' \\
    car.lower() & 'lightning mcqueen from cars' \\
    car.title() & 'Lightning Mcqueen From Cars' \\
    car.capitalize() & 'Lightning mcqueen from cars' \\
    car.swapcase() & 'lIGHTNING mCqUEEN FROM cARS'
  \end{table}
\end{frame}


%
% Division / Joining
%
\begin{frame}
  \frametitle{Division}
\end{frame}


%
% Search
%
\begin{frame}
  \frametitle{Division}
\end{frame}


%
% Replace
%
\begin{frame}
  \frametitle{Division}
\end{frame}


%
% Testing
%
\begin{frame}
  \frametitle{Testing}
\end{frame}


%
% String format methods
%
\begin{frame}
  \frametitle{String format methods}
\end{frame}


%
% format() built-in
%
\begin{frame}
  \frametitle{Formatting Values} 
\end{frame}


%
% Leading Letters
%
\begin{frame}
  \frametitle{Letters Before Strings}
  \begin{itemize}
    \item Unicode (Python 2)
      \begin{itemize}
        \item Strings are unicode by default in Python 3
        \item Ex.: \lstinline{u'string'}
      \end{itemize}

    \item Bytes
      \begin{itemize}
        \item An unencoded sequence of bytes
        \item Ex.: \lstinline{b'string'}
      \end{itemize}

    \item Raw
      \begin{itemize}
	\item Don't have to escape \textbackslash
        \item Ex.: \lstinline{r'C:\\Users'}
      \end{itemize}

    \item Format (Python 3.6) 
      \begin{itemize}
	\item Can refer to variables in-string
        \item Ex.: \lstinline{name = 'Daniel'; print(f"Hello, \{name\}")}
      \end{itemize}
    \end{itemize}
  \end{frame}

%
% Ways to Format
%
\begin{frame}
  \frametitle{Formatting Variables}
  \begin{itemize}
    \item C-Style
      \begin{itemize}
        \item \% places a new variable in a string
	\item Letter means variable's type
	\item Values between specify length
        \item Ex.: \lstinline{"\%s are \%.2f" \% (item, cost)}
      \end{itemize}

    \item Positional
      \begin{itemize}
        \item Sets of brackets place a new variable in a string
	\item Variables placed in \texttt{str.format} argument order
	\item Values after (optional) colon specify length
        \item Ex.: \lstinline{"\{\} are \{:.2\}".format(item, cost)}
	\item If temporary names given, order arbitrary
        \item Ex.: \lstinline{"\{x\} are \{y:.2\}".format(x=item, y=cost)}
        \item Ex.: \lstinline{"\{item\} are \{cost:.2\}".format(cost=cost, item=item)}
       \end{itemize}

    \item Format String (Python 3.6) 
      \begin{itemize}
        \item Ex.: \lstinline{f'\{item\} are \{cost:.2\}'}
      \end{itemize}
    \end{itemize}
  \end{frame}
\end{document}
