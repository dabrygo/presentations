\documentclass{beamer}

\usepackage{color}
\usepackage{graphicx}
\usepackage{hyperref}
\usepackage{listings}
\usepackage{minted}


\title{IntroPy}
\subtitle{Strings}


\begin{document}
\frame{\titlepage}

%## Strings
%  * API
% * Formatting
%   * `%`
%   * `.format()`
%   * `f'{}'`
%   * template?
%   * Numbers
% * b''
% * r''
% * Unicode
% * str
% TODO Bytes in 2.7 is 'str' but 'bytes' in 3.6

%
% Properties
%
\begin{frame}
  \frametitle{Properties}
  \begin{itemize}
    \item Immutable
    \item Iterable
    \item Indexable
  \end{itemize}
\end{frame}


%
% Making a String
%
\begin{frame}
  \frametitle{Making a String}
  \begin{itemize}
    \item Single- or double-quotes
      \begin{itemize}
        \item \texttt{`hello world'} 
	\item \texttt{"hello world"}
      \end{itemize}

    \item Three-quote pairs for multiline text
%    \inputminted{python}{multiline.py}

    \item Can nest different quotes inside each other
      \begin{itemize}
        \item \texttt{`Say "hello"'} 
	\item \texttt{"`hello'"}
      \end{itemize}
  \end{itemize}
\end{frame}


%
% Escaping
%
\begin{frame}
  \frametitle{Special Characters and Escaping}
  \begin{itemize}
    \item Special characters start with \textbackslash, e.g.: 
      \begin{itemize}
        \item \texttt{"\textbackslash n"} (newline)
        \item \texttt{"\textbackslash t"} (tab)
	\item \texttt{"\textbackslash " "} or \texttt{`\textbackslash ` '} 
      \end{itemize}

    \item \textbackslash \textbackslash~ escapes backslash 
      \begin{itemize}
        \item \texttt{`C:\textbackslash\textbackslash Users'}
      \end{itemize}

    \item Lines are continued if there's nothing after \textbackslash 
      \begin{itemize}
        \item \texttt{```This string is\textbackslash\\two lines long!'''}
      \end{itemize}
  \end{itemize}
\end{frame}


%
% Capitalization
%
\begin{frame}
  \frametitle{Case}
  Given \texttt{car = `Lightning McQueen in Cars'}
  \begin{table}
    \begin{tabular}{l | l | l}
    Function & Output & Test \\
    \hline
    car.upper() & `LIGHTNING MCQUEEN IN CARS' & str.isupper() \\
    car.lower() & `lightning mcqueen in cars' & str.islower() \\
    car.title() & `Lightning Mcqueen In Cars' & str.istitle() \\
    car.capitalize() & `Lightning mcqueen in cars' & N/A \\
    car.swapcase() & `lIGHTNING mCqUEEN IN cARS' & N/A \\
    \end{tabular}
  \end{table}
  ``der Flu\ss".casefold() == ``der Fluss".casefold()
\end{frame}


%
% Padding 
%
\begin{frame}
  \frametitle{Padding}
  \textit{str.ljust(width[, fillchar]) -\textgreater~ str}
  \begin{table}
    \begin{tabular}{l | l}
    Function & Output \\
    \hline
    ``7".zfill(3) & `007' \\
    ``Python".ljust(1) & `Python' \\
    ``Python".ljust(12, '*') & `Python******' \\
    ``Python".rjust(12, '*') & `******Python' \\
    ``Python".center(12, '*') & `***Python***' \\
    ``Python".center(13, '*') & `****Python***'
    \end{tabular}
  \end{table}
\end{frame}


%
% Stripping
%
\begin{frame}
  \frametitle{Stripping}
  \textit{str.strip([chars]) -\textgreater~ string}

  \texttt{line = "\textbackslash t   hello world\textbackslash t\textbackslash t"}

  \begin{table}
    \begin{tabular}{l | l}
    Function & Output \\
    \hline
    line.lstrip()     & \lstinline{``hello world\\t\\t"} \\
    line.rstrip()     & \lstinline{``\\t   hello world"} \\
    line.strip()      & \lstinline{``hello world"}       \\
    line.strip('\textbackslash t') & \lstinline{``   hello world"}
    \end{tabular}
  \end{table}
\end{frame}


%
% Division / TODO Joining
%
\begin{frame}
  \frametitle{Splitting and Joining}

  \textit{str.split(sep=None, maxsplit=-1) -\textgreater~ list of strings}

  \texttt{names = `Chapman Cleese\textbackslash tIdle      Palin'}

  \texttt{csv = `Chapman,Cleese,Idle,Palin'}

  \texttt{lines = `Chapman\textbackslash nCleese\textbackslash nIdle\textbackslash nPalin'}

  \texttt{troupe = [`Chapman', `Cleese', `Idle', `Palin']}
  
  \begin{table}
    \begin{tabular}{l | l}
    Function & Output \\
    \hline
    names.split() & ['Chapman', 'Cleese', 'Idle', 'Palin'] \\
    csv.split(sep=',') & ['Chapman', 'Cleese', 'Idle', 'Palin'] \\
    csv.split(sep=',', maxsplit=1) & ['Chapman', 'Cleese,Idle,Palin'] \\
    csv.rsplit(sep=',', maxsplit=2) & ['Chapman,Cleese', 'Idle', 'Palin'] \\
    lines.splitlines() & ['Chapman', 'Cleese', 'Idle', 'Palin'] \\
    \hline
    ' + '.join(troupe) & 'Chapman + Cleese + Idle + Palin'
    \end{tabular}
  \end{table}
\end{frame}


%
% Slicing
%
\begin{frame}
  \frametitle{Slicing}

  Given \texttt{text = 'words'}

  \begin{itemize}
    \item \texttt{text[start:stop:step]} extracts a substring by indices
    \item \texttt{start} is inclusive, \texttt{stop} is exclusive
    \item Omitted arguments fall back to defaults

      \texttt{text[0:len(text):1] == text[:::] == text}

    \item Good practice to use no more than two of \texttt{start, stop, step} at a time

    \item Negatives mean backward movement
    
      Reverse word: \texttt{text[::-1] == `sdrow'}

      Next to last letter: \texttt{text[:-1] == `word'}

      Next to last letter: \texttt{text[-1] == `s'}

      Last three letters: \texttt{text[-3:] == `rds'}
  \end{itemize}

\end{frame}

%
% Search
%
\begin{frame}
  \frametitle{Search}
  All have form \textit{str.search(substring[, start[, end]])}
  
  \texttt{word = ``floccinaucinihilipilification"}

  \begin{table}
    \begin{tabular}{l | l}
    Function & Output \\
    \hline
    word.count('i') & 9 \\
    word.count('in') & 2 \\
    word.count('i', 10) & 8 \\
    word.count('i', 0, 6) & 1 \\
    \hline
    word.startswith('f') & True \\
    word.endswith('N') & False \\
    \hline
    word.find('il') & 15 \\
    word.find('fjord') & -1 \\
    \hline
    word.index('il') & 15 \\
    word.index('fjord') & ValueError
   \end{tabular}
  \end{table}
\end{frame}


%
% Replace
%
\begin{frame}
  \frametitle{Replace}
  tab = '\\t'
  arthur = '1... 2... 5'
  song = 'apples and bananas'
  \begin{table}
    \begin{tabular}{l | l}
    Function & Output \\
    \hline
    tab.expandtabs() & '        ' \\
    tab.expandtabs(tabsize=2) & '  ' \\
    \hline
    arthur.replace('5', '3') & '1... 2... 3'\\
    arthur.replace('6', '3') & '1... 2... 5'\\
    arthur.replace('5', '3', 2) & '1... 2... 3'\\
    song.replace('a', 'oo') & 'ooples oond boonoonoos'\\
    song.replace('a', 'oo', 2) & 'ooples oond bananas'\\
   \hline
    % TODO
    str.encode()
   \end{tabular}
  \end{table}

  \texttt{\textgreater \textgreater \textgreater fudd = str.maketrans('LlRr', 'WwWw')}

  \texttt{\textgreater \textgreater \textgreater ``Let's play, rabbit".translate(fudd)}

  \textit{Wet's pway, wabbit}
\end{frame}


%
% Testing
%
\begin{frame}
  \frametitle{Testing}
  \begin{table}
    \begin{tabular}{l | l}
      Function & Output \\
      \hline
      `abc123'.isalnum() & True \\
      `abc'.isalpha() & True \\
      `3.14'.isdecimal() & True \\
      `123'.isdigit() & True \\
      `my\_variable1'.isidentifier() & True \\
      `$\frac{1}{2}$'.isnumeric() & True \\
      `\textbackslash n'.isprintable() & False \\
      `\textbackslash n'.isspace() & True 
    \end{tabular}
  \end{table}
\end{frame}


%
% format() built-in
%
\begin{frame}
  \frametitle{Formatting Values} 
\end{frame}


%
% Leading Letters
%
\begin{frame}
  \frametitle{Letters Before Strings}
  \begin{itemize}
    \item Unicode (Python 2)
      \begin{itemize}
        \item Strings are unicode by default in Python 3
        \item Ex.: \lstinline{u'string'}
      \end{itemize}

    \item Bytes
      \begin{itemize}
        \item An unencoded sequence of bytes
        \item Ex.: \lstinline{b'string'}
      \end{itemize}

    \item Raw
      \begin{itemize}
	\item Don't have to escape \textbackslash
        \item Ex.: \lstinline{r'C:\\Users'}
      \end{itemize}

    \item Format (Python 3.6) 
      \begin{itemize}
	\item Can refer to variables in-string
        \item Ex.: \lstinline{name = 'Daniel'; print(f"Hello, \{name\}")}
      \end{itemize}
    \end{itemize}
  \end{frame}

%
% Ways to Format
%
\begin{frame}
  \frametitle{Formatting Variables}
  \begin{itemize}
    \item C-Style
      \begin{itemize}
        \item \% places a new variable in a string
	\item Letter means variable's type
	\item Values between specify length
        \item Ex.: \lstinline{"\%s are \%.2f" \% (item, cost)}
      \end{itemize}

    \item Positional
      \begin{itemize}
        \item Sets of brackets place a new variable in a string
	\item Variables placed in \texttt{str.format} argument order
	\item Values after (optional) colon specify length
        \item Ex.: \lstinline{"\{\} are \{:.2\}".format(item, cost)}
	\item If temporary names given, order arbitrary
        \item Ex.: \lstinline{"\{x\} are \{y:.2\}".format(x=item, y=cost)}
        \item Ex.: \lstinline{"\{item\} are \{cost:.2\}".format(cost=cost, item=item)}
      \end{itemize}
  \end{itemize}
\end{frame}

%
% Formatting
%
\begin{frame}
  \frametitle{Formatting}
  \begin{itemize}
    \item Can refer to position:
    \begin{itemize}
      \texttt{"\{1\} is \{0\}".format('blue', 'The car')}
    \end{itemize}
    \item Can refer to attributes: 
    \begin{itemize}
      \item \texttt{nm = (35, -106)}
      \item \texttt{"Lat: \{loc[0]\}, lon: \{loc[1]\}".format(loc=nm)}
      \item \texttt{"Lat: \{0[0]\}, lon: \{0[1]\}".format(nm)}
    \end{itemize}
    \item Can refer to fields:
    \begin{itemize}
      \item \texttt{import collections}
      \item \texttt{LatLon = collections.namedtuple('LatLon', 'lat lon')}
      \item \texttt{latlon = LatLon(36, -106)}
      \item \texttt{"Lat: \{loc.lat\}, lon: \{loc.lon\}" .format(loc=latlon)}
    \end{itemize}
  \end{itemize}
\end{frame}

%
% Python 3.6 formatting
%
\begin{frame}
 \frametitle{Python 3.6 Formatting}
 \begin{itemize}
   \item Ex.: \lstinline{f'\{1\} are \{cost:.2\}'}
 \end{itemize}
\end{frame}

\end{document}
