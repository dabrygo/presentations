\documentclass{beamer}

%\usepackage{color}
%usepackage{graphicx}
%usepackage{hyperref}
\usepackage{listings}
%usepackage{minted}


\title{IntroPy}
\subtitle{Containers}

\begin{document}

\maketitle

\begin{frame}
  \textbf{What are Containers?}

  \begin{itemize}
    \item Can hold items of same/different types
    \item Can use \texttt{in} to test if object in container
  \end{itemize}

  Sometimes we can:
  \begin{itemize}
    \item Use slices [::] like we did for \texttt{str}
    \item Use \texttt{len} to count items
    \item Nest them inside each other
  \end{itemize}
\end{frame}

\begin{frame}
  \frametitle{tuple}

  \textbf{Purpose}: Unchangeable (usually small) ordered collection of items

  \textbf{Syntax}: 
  \begin{itemize}
    \item \texttt{(`abc',)} 
    \item \texttt{(`abc', `abc', 123)} 
    \item \texttt{x = `abc', `abc', 123} or \texttt{return `abc', `abc', 123}
  \end{itemize}

  \textbf{Quirks}
  \begin{itemize}
    \item No ``augmented'' operations (+=, -=, etc.)
    \item Has a ``named'' version (collections.namedtuple)
  \end{itemize}
\end{frame}

\begin{frame}[fragile]
  \frametitle{list}

   \textbf{Purpose}: Changeable, ordered collection of items

   \textbf{Syntax}: 
   \begin{itemize}
     \item \texttt{[`abc']} 
     \item \texttt{[`abc', `abc', 123]}
     \item \texttt{list((`abc', `abc', 123))} turns into \texttt{[`abc', `abc', 123]}
   \end{itemize}

  \textbf{Quirks}
  \begin{itemize}
    \item Like a ``stack'': \texttt{list.append()} and \texttt{list.pop()}
    \item += works with \texttt{str}
    \item Can ``replace'' slices with different size slices 
    \begin{lstlisting}
      >>> x = [1, 2, 3, 4]
      >>> x[1:3] = [6]
      >>> x
      [1, 6, 4]
    \end{lstlisting}
  \end{itemize}
\end{frame}

\begin{frame}
  \frametitle{Builtins}

  \begin{itemize}
    \item \texttt{any}
    \item \texttt{all}
    \item \texttt{max}
    \item \texttt{min}
    \item \texttt{reversed}
    \item \texttt{sum}
  \end{itemize}
\end{frame}

\begin{frame}
  \frametitle{set} 

  \textbf{Purpose}: Unique items, quick look-ups, like a mathematical set

  \textbf{Syntax}: 
  \begin{itemize}
    \item \texttt{\{`abc'\}} 
    \item \texttt{\{`abc', 1\}}
    \item \texttt{set([`abc', `abc', 123])} turns into \texttt{\{`abc', 123\}}
  \end{itemize}

  \textbf{Quirks}
  \begin{itemize}
    \item Can use symbols or functions for mathematical operations
      \begin{itemize}
        \item \& is \texttt{set.intersection}
	\item $\vert$ is \texttt{set.union}
	\item \string^ is \texttt{set.symmetric\_difference}
	\item $-$ is \texttt{set.difference}
      \end{itemize}
  \end{itemize}
\end{frame}

\begin{frame}
  \frametitle{dict}
 
  \textbf{Purpose}: List with a customizable (non-number) index

  \textbf{Syntax}: 
  \begin{itemize}
    \item \texttt{\{`abc': 123\}} 
    \item \texttt{\{`abc': 1, `def': 1, `ghi': 2\}}
    \item \texttt{dict([(`abc', 1), (`def', 1), (`ghi', 2)])}
          turns into \texttt{\{`abc': 1, `def': 1, `ghi': 2\}}
  \end{itemize}

  \textbf{Quirks}
  \begin{itemize}
    \item Ordered in Python 3.6 (keys returned in order inserted)
    \item Has a formally ordered version (collections.ordereddict)
  \end{itemize}
\end{frame}

\begin{frame}
  \frametitle{Unpacking}

  \begin{itemize}
  \item Assign values directly to variables with one assignment
    \begin{itemize}
      \item \texttt{a, b, c = 1, 2, 3} is the same as \texttt{a = 1; b = 2; c = 3}
      \item \texttt{a, b, c = [1, 2, 3]} is the same thing
     \end{itemize}
  \item * packs everything else into that variable
    \begin{itemize}
      \item \texttt{head, *body = [1, 2, 3]} $\rightarrow$ \texttt{head=1; body=[2, 3]}
      \item \texttt{*body, tail = [1, 2, 3]} $\rightarrow$ \texttt{body=[1, 2]; tail=3}
      \item \texttt{head, *body, tail = [1, 2, 3, 4]} $\rightarrow$ \texttt{head=1; body=[2, 3]; tail=4}
     \end{itemize}
  \end{itemize}
\end{frame}

\begin{frame}
  \frametitle{Comprehensions}

  \textbf{Purpose}: Build a container concisely

  \begin{itemize}
    \item tuple: \texttt{tuple(i for i in range(3))}
    \item list: \texttt{[x for x in range(3)]}
    \item set: \texttt{\{x for x in range(3)\}}
    \item dict: \texttt{\{k:v for k,v in [(`a',1), (`b',2), (`c',3)]\}}
  \end{itemize}

\end{frame}

\begin{frame}[fragile]
  \frametitle{generator}
 
  \textbf{Purpose}: Use with large/unending collections 

  \textbf{Syntax}: 
  \begin{itemize}
  \item \texttt{(i for i in range(10000))}

  \item 

  \begin{lstlisting}
    def evens():
      n = 0
      while True:
        n += 2
	yield n
  \end{lstlisting}

  \end{itemize}

  \textbf{Quirks}
  \begin{itemize}
    \item ``Consumed'' on use
  \end{itemize}
\end{frame}

\begin{frame}
  \frametitle{map}

  \textbf{Purpose}: Apply a function to every item in a collection

  \textbf{Syntax}: 
  \begin{itemize}
    \item \texttt{map(int, [`1', `2', `3'])} turns \texttt{str} to \texttt{int}
  \end{itemize}

  \textbf{Quirks}
  \begin{itemize}
    \item Returns a \texttt{map} object, but can be cast to \texttt{list}
    \item \texttt{[int(x) for x in (`1',`2',`3')]}
  \end{itemize}
\end{frame}

\begin{frame}
  \frametitle{filter}

  \textbf{Purpose}: Choose elements from a collection

  \textbf{Syntax}: 
  \begin{itemize}
    \item \texttt{filter(lambda x: x \% 2 == 1, range(10))} is odd numbers less than 10
  \end{itemize}

  \textbf{Quirks}
  \begin{itemize}
    \item Returns a \texttt{filter} object, but can be cast to \texttt{list}
  \end{itemize}
\end{frame}

\begin{frame}
  \frametitle{zip}

  \textbf{Purpose}: Combine multiple sequences into one

  \textbf{Syntax}: 
  \begin{itemize}
    \item \texttt{zip([`a',`b',`c'],(1,2,3))} creates tuples \texttt{(`a',1),(`b',2),(`c',3)}
  \end{itemize}

  \textbf{Quirks}
  \begin{itemize}
    \item Returns a \texttt{zip} object, but can be cast to \texttt{list}
  \end{itemize}
\end{frame}


\end{document}
