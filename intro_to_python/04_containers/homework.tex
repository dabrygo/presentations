\documentclass{article}

\begin{document}
  
  \begin{enumerate}
  \item \textbf{Tuple}. Tuples are usually used for closely related
  data that can be easily unpacked into variables or referenced by
  position. Examples:
  \begin{itemize}
    \item \texttt{point = (1, 2)}
    \item \texttt{huey, luey, duey = ('huey', 'luey', 'duey')}
    \item \texttt{lla = (-35, 105, 2000); latitude = lla[0]}
  \end{itemize}
  Can you think of examples where you would use a tuple?

  \item \textbf{List (1)}. Use \texttt{help} to figure out the difference between
  \texttt{list.remove} and \texttt{list.pop}.

  \item \textbf{List (2)}. Given the lists:
  \begin{verbatim}
    first = ['William', 'Warren', 'John', 'Franklin', 'John']
    middle = ['Henry', 'Gamaliel', 'Calvin', 'Delano', 'Fitzgerald']
    last = ['Harrison', 'Harding', 'Coolidge', 'Roosevelt', 'Kennedy']
  \end{verbatim}
  Use a loop to print out the names in the form: \texttt{Last, First M.}
  Can you do it with and without \texttt{zip}? 

  \item \textbf{Set}. You are given the words \textit{sassafras}
  and \textit{fracases}. Use sets to programmatically determine:
  \begin{enumerate}
    \item The \textit{number} of unique letters in each word (remember to use \texttt{len})
    \item Which and how many letters both words have in common 
    \item Which and how many letters either word (but not both) have
    \item Which and how many letters \textit{sassafras} has that \textit{fracases} doesn't
    \item Which and how many letters \textit{fracases} has that \textit{sassafras} doesn't
  \end{enumerate}
  \textsc{Note}: We can cast a \texttt{str} to a \texttt{set} by using \texttt{set('aab')}

  \item \textbf{Dict}. The \texttt{collections} module has a special \texttt{dict}
  called \texttt{Counter}. The code:
  \begin{verbatim}
  from collections import Counter

  c = Counter("aBacAb")
  print(c)
  \end{verbatim}
  displays 
  
  \begin{center}
    \texttt{Counter(\{'a': 2, 'B': 1, 'c':1, 'A': 1, 'b': 1\})}
  \end{center}
  
  the count of each character (case-insensitive) in the string. We can
  access the count of `a' by doing \texttt{c['a']}.

  In genetic biology, the GC content of a DNA/RNA strand is used to classify
  chromosomes. It is a percentage, the count of G's and C's of a chromosome
  divided by its total length. Use \texttt{counter} and \texttt{len} to find
  the GC content of a DNA (string of G,A,T,C) or RNA (string of G,A,U,C) strand
  of your choice.

  \item Recall the standard collection types:
  \begin{itemize}
    \item list
    \item tuple/namedtuple
    \item dict
    \item set
    \item generator
  \end{itemize}
  You're making a program for a public library.
  Which data structure(s) might you use when the library wants:

  \begin{enumerate}
    \item A book to be a title, author, and ISBN
    \item To know how many copies of each book they have
    \item To check if a book is in the library/inventory
    \item To search for text in books
    \item To look up books by Dewey decimal number
    \item To know what books are checked out on a given patron's card
    \item To know what the most popular/checked-out books of the week were
    \item To make a "first come, first served" hold list 
    \item To find which patrons have copies of some book
    \item To automatically limit the number of copies of a popular book 
          that one patron/family can check out
    \item To limit how long books (especially popular ones) can be checked out
  \end{enumerate}

  \textsc{Note}: These questions aren't meant to have objective answers.
  There more to get you into the mode of thinking about the pros and cons
  of the different collection types.

  \item In lecture we noticed a list is like a ``stack''. A common example is
  cafeteria trays, where the topmost tray is both the last item stacked and the
  first tray to come off (Last In First Out). Another paradigm is the
  ``queue,'' where the first item in is the first item out (First In First
  Out), like customers in line at a bank. The deque (double-ended queue) from
  the \texttt{collections} module has methods that let it behave like both a
  ``stack'' and a ``queue''. Does knowing about this data structure change any
  of your answers to the library question?
  \end{enumerate}
 
\end{document}
