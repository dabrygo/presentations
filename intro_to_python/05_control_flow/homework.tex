\documentclass{article}

\usepackage{graphicx}

\begin{document}
  \begin{enumerate}
    \item ``Rock, Paper, Scissors" is a game where two players
      choose an item from the game's title and compare results. 
      The winning results are:

      \begin{center}
        \includegraphics[width=\textwidth]{rps_win.png}
      \end{center}
      Furthermore, when the players' choices are identical, the result is a draw.
      Program this logic in Python.

      \textsc{idea}: Use \texttt{random.choice} to randomly choose an element
      from a collection.

      \textsc{hint}: We can compare strings a user enters using \texttt{input} and $==$:
      \begin{verbatim}name = input("Enter your name: ")
	if name == "daniel":
	  print("you are daniel")
	else:
	  print("not daniel")
      \end{verbatim}

    \item The Fibonacci sequence is defined as $F_n = F_{n-2} + F_{n-1}$, where
    each element is the sum of the two previous and the first two elements are
    $F_1 = 1$ and $F_2 = 1$. The sequence starts like:

    \begin{center}
      1, 1, 2, 3, 5, 8, 13, 21, 34, 55, \ldots
    \end{center}

    Complete this function to find the nth Fibonacci number:
    \begin{verbatim}
      def fibonacci(n):
        return 
    \end{verbatim}

  \item Make a number-guessing game where the player has to find a number
  the computer chooses from 1 to 100 using \texttt{random.randint}. As an
  aid to the player, print when their guess is too high or too low.


  \item This code is for an adventure game:
  \begin{verbatim}
  inventory = {'bread': 5, 'sword': 1, 'gold': 42}

  def take(quantity, item):
    if not any(inventory.values()):
      raise Exception("I'm not carrying anything")
    amount = inventory[item]
    if quantity > amount:
      raise ValueError(f"Can't take {quantity} {item}; I only have {amount}")
    inventory[item] = amount - quantity
  \end{verbatim}
  
  Using \texttt{try-except} handle the following sequence of calls gracefully, 
  so that the inventory is "empty" (zeros for values) at the end:
  \begin{verbatim}
    take(2, 'sword')
    take(5, 'bread')
    take(1, 'kangaroo')
    take(1, 'sword')
    take(41, 'gold')
    take(2, 'gold')
    take(1, 'gold')
    take(1, 'gold')
    print(inventory)
  \end{verbatim}
  \end{enumerate}
\end{document}
