\documentclass{beamer}

%\usepackage{color}
%usepackage{graphicx}
%usepackage{hyperref}
\usepackage{listings}
%usepackage{minted}


\title{IntroPy}
\subtitle{Control Flow}

\begin{document}

\maketitle

\begin{frame}
  \frametitle{Control Flow in Python}

  \begin{enumerate}
    \item Conditional: \texttt{if-elif-else} 
    \item Loops: \texttt{while} and \texttt{for}
    \item Error handling: \texttt{try-except-finally}
  \end{enumerate}

  \textbf{NOT Keywords in Python}
  \begin{enumerate}
    \item \texttt{do-while}
    \item \texttt{switch-case-default}
    \item \texttt{until}
  \end{enumerate}
\end{frame}


\begin{frame}[fragile]
  \frametitle{General Syntax}

  \begin{enumerate}
    \item Blocks of related code are called a \textit{suite}
    \item Indentation groups related statements
    \item But indentation need only be shared for same \textit{suite} 
    \item (Still good practice to be consistent...)
    \item PEP 8 recommends expanding tabs and 4 spaces for each new level
  \end{enumerate}

  \textbf{Form}
  \begin{lstlisting}
    keyword expression:
      suite
  \end{lstlisting}
\end{frame}

\begin{frame}[fragile]
  \frametitle{\texttt{If-Elif-Else}}

  \begin{enumerate}
    \item Each part introduces its own suite
    \item Parentheses unnecessary around conditionals
  \end{enumerate}

  \textbf{Form}
  \begin{lstlisting}
    a = 1
    if a < 0:
      print("negative")
    elif a > 0:
      print("positive")
    else:
      print("zero")
  \end{lstlisting}
\end{frame}

\begin{frame}[fragile]
  \frametitle{\texttt{While}}

  \begin{enumerate}
    \item \texttt{while} loops have an optional \texttt{else}
    \item \texttt{else} runs when condition \textit{evaluated} to be false
    \item This includes when loop does not run at all
    \item (Use of \texttt{else} here is rare and discouraged...)
  \end{enumerate}

  \textbf{Form}
  \begin{lstlisting}
    a = 0
    b = 1
    while a < 10:
      print(a)
      a += 1
    else:
      print(b)
  \end{lstlisting}
\end{frame}

\begin{frame}[fragile]
  \frametitle{\texttt{For}}

  \begin{enumerate}
    \item Also have an optional \texttt{else}
    \item Tend to be used on collections, as \texttt{for-each} loops
    \item But can behave like standard \texttt{for} loops using \texttt{range}
  \end{enumerate}

  \textbf{Form}
  \begin{lstlisting}
    for odd in range(1, 11, 2):
      print(odd)
    else:
      print("end")
  \end{lstlisting}
\end{frame}

\begin{frame}[fragile]
  \frametitle{\texttt{For}}

  \begin{enumerate}
    \item Can iterate over more than one ``key'', like index and item from \texttt{enumerate}
  \end{enumerate}

  \textbf{Example 1}
  \begin{lstlisting}
    for i, letter in enumerate(`abc', start=1):
      print(f`Letter #{i}: {letter}')
  \end{lstlisting}

  \textbf{Example 2}
  \begin{lstlisting}
    d = {'a': 1, 'b': 2, 'c': 3}
    for key, value in d.items():
      print(f`Letter #{value}: {key}')
  \end{lstlisting}
\end{frame}


\end{document}
